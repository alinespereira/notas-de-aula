\section[Variáveis aleatórias discretas]{Distribuição de probabilidade de variáveis aleatórias discretas}

\begin{definition}
    A distribuição de probabilidade de uma variável aleatória (\va) discreta
    $X$ é uma função que atribui a cada valor de $X$ sua respectiva
    probabilidade.
\end{definition}

\begin{example}
    Lançamento de uma moeda com $X(\{\text{coroa}\}) = 0$ e $X(\{\text{cara}\}) = 1$.
    Tomamos $p_C \in (0, 1)$.

    \begin{center}
        \begin{tabular}{cc}
            \toprule
            $x$ & $\Prob(X=x)$ \\
            \midrule
            $0$ & $1 - p_C$ \\
            $1$ & $p_C$ \\
            \midrule
            Soma & 1 \\
            \bottomrule
        \end{tabular}
    \end{center}
\end{example}

\begin{example}[Continuação do \cref{exp:ch02-va-moeda}]
    Temos:
    \begin{center}
        \begin{tabular}{cc}
            \toprule
            $x$ & $\Prob(X=x)$ \\
            \midrule
            $0$ & $\sfrac{1}{4}$ \\
            $1$ & $\sfrac{1}{2}$ \\
            $2$ & $\sfrac{1}{4}$ \\
            \midrule
            Soma & 1 \\
            \bottomrule
        \end{tabular}
    \end{center}

    Se tivermos $p_C$ qualquer, a distribuição passa a ser:
    \begin{center}
        \begin{tabular}{cc}
            \toprule
            $x$ & $\Prob(X=x)$ \\
            \midrule
            $0$ & $(1 - p_C)^2$ \\
            $1$ & $\color{detail-color}{1 - (1 - p_C)^2 - p_C^2}$ \\
            $2$ & $p_C^2$ \\
            \midrule
            Soma & 1 \\
            \bottomrule
        \end{tabular}
    \end{center}
\end{example}

\begin{example}
    No lançamento de um dado de seis faces, a probabilidade de cada face
    é proporcional ao número de pontos na face. A \va\ $X$ representa 
    o resultado do lançamento, sendo que $X \in \{1,2,3,4,5,6\}$.
    Apresente a distribuição de probabilidade de $X$.

    \bigskip
    Para $x \in \{1,2,3,4,5,6\}$, temos que
    \begin{align*}
        \Prob(X = x) &= kx
    \end{align*}
    e
    \begin{align*}
        \sum_{x=1}^6 \Prob(X = x) &= 1
    \end{align*}

    Portanto,
    \begin{align*}
        \sum_{x=1}^6 kx &= 1 \\
        \implies k \sum_{x=1}^6 x &= 1 \\
        \therefore k &= \frac{1}{\sum_{x=1}^6 x}
    \end{align*}
    ou seja,
    \begin{align*}
        k &= \frac{1}{\sum_{x=1}^6 x} = \frac{1}{21}
    \end{align*}

    Portanto, a distribuição de $X$ é:
    \begin{center}
        \begin{tabular}{cc}
            \toprule
            $x$ & $\Prob(X=x)$ \\
            \midrule
            $1$ & $\sfrac{1}{21}$ \\
            $2$ & $\sfrac{2}{21}$ \\
            $3$ & $\sfrac{3}{21}$ \\
            $4$ & $\sfrac{4}{21}$ \\
            $5$ & $\sfrac{5}{21}$ \\
            $6$ & $\sfrac{6}{21}$ \\
            \midrule
            Soma & 1 \\
            \bottomrule
        \end{tabular}
    \end{center}
\end{example}

\begin{obs}
    A distribuição de probabilidade de uma \va\ discreta também é chamada
    de \textbf{função massa de probabilidade}.
\end{obs}

\begin{example}
    Um dado de seis faces é lançado. A \va\ $X$ representa o número do lançamento
    em que ocorre o resultado $\{6\}$ pela primeira vez. Neste caso,
    $X \in \{1,2,\cdots\} = \Nat$. Apresente a função de probabilidade da \va $X$.

    \bigskip
    Supomos que os resultados dos lançamentos são eventos independentes.

    Digamos que $X = n$, $n \in \Nat$, dignifica que $n -1$ resultados
    são do conjunto $\{1,2,3,4,5\}$ e o n-ésimo resultado é $\{6\}$.
    Logo:
    \begin{align*}
        \Prob(X = n) &= \left(\frac{5}{6}\right)^{n-1} \cdot \frac{1}{6}
    \end{align*}

    Sendo $A = \{6\}$, a sequência de eventos independentes é
    \begin{align*}
        \underbrace{\stcomp{A}, \stcomp{A}, \cdots, \stcomp{A}}_
            {n - 1\ \text{vezes}}, A 
    \end{align*}

    \begin{center}
        \begin{tikzpicture}
            \begin{axis}[
                unbounded coords=jump,
                ymajorgrids=true,
                major grid style={dashed},
                axis x line=middle,
                axis y line=middle,
                xmin=0, xmax=10,
                ymin=0, ymax=0.2,
                xtick={1,2,...,6},
                ytick={
                    (5/6)^5*1/6,
                    (5/6)^4*1/6,
                    (5/6)^3*1/6,
                    (5/6)^2*1/6,
                    5/6*1/6,
                    1/6
                },
                yticklabels={
                    $\left(\sfrac{5}{6}\right)^5 \cdot \sfrac{1}{6}$,
                    \empty,
                    $\left(\sfrac{5}{6}\right)^3 \cdot \sfrac{1}{6}$,
                    \empty,
                    $\sfrac{5}{6} \cdot \sfrac{1}{6}$,
                    $\sfrac{1}{6}$,
                },
                xlabel={$n$},
                ylabel={$\Prob(X =n)$},
                y label style={anchor=south},
                x label style={anchor=west},
            ]
            
            \addplot+[
                ycomb,
                dashed,
                samples=6, 
                domain=1:6,
                blue]
                {(1/6)*(5/6)^(x-1)};

            \node[blue, ultra thick] at (8, {(5/6)^5*1/6*1/2}) {$\cdots$};

            \end{axis}
            \label{fig:ch02-geom-dado-seis}
        \end{tikzpicture}
    \end{center}

    Vamos veificar o axioma \labelcref{it:ch01-axioma-3} da \cref{def:axiomatica}. 
    Calculamos:
    \begin{align*}
        \sum_{n=1}^\infty \Prob(X = n)
        &= \sum_{n=1}^\infty \left(\frac{5}{6}\right)^{n-1} \cdot \frac{1}{6} \\
        &= \frac{1}{6} \cdot \sum_{n=1}^\infty \left(\frac{5}{6}\right)^{n-1} \\
        &= \frac{1}{6}
            \cdot \underbrace{
                \left[1 + \frac{5}{6} + \left(\frac{5}{6}\right)^2 + \cdots\right]
            }_{\text{Soma da PG infinita de razão}\ \sfrac{5}{6}}\\
        &= \frac{1}{6} \cdot \frac{1}{1 - \frac{5}{6}} \\
        &= \frac{1}{6} \cdot 6 = 1
    \end{align*}
\end{example}