\section{Variáveis aleatórias}

\begin{example}[Lancamento de duas moedas]\label{exp:ch02-va-moeda}
    Registramos o número de resultados "cara".

    Resultados: cara ($C$) ou coroa ($\overline{C}$).

    \begin{center}
        \begin{tabular}{cc}
            \toprule
            Evento                          & Número de caras \\
            \midrule
            $\{C,C\}$                       & 2 \\
            $\{C,\overline{C}\}$            & 1 \\
            $\{\overline{C},C\}$            & 1 \\
            $\{\overline{C},\overline{C}\}$ & 0 \\
            \bottomrule
        \end{tabular}
    \end{center}
\end{example}

\begin{definition}
    Uma \textbf{variável aleatória} é uma função
    definida no espaço amostral e com valores reais.
    \begin{align*}
        X: S \to \R
    \end{align*}
    \begin{notation}
        Letras maiúsculas, últimas do alfabeto (geralmente).
    \end{notation}
\end{definition}

\begin{example}[Continuação do \cref{exp:ch02-va-moeda}]
    \begin{align*}
        X(\{C,C\})                       &= 2 \\
        X(\{C,\overline{C}\})            &=
        X(\{\overline{C},C\})            = 1 \\
        X(\{\overline{C},\overline{C}\}) &= 0
    \end{align*}
\end{example}

\begin{example}\label{exp:ch02-va-pedagio}
    $X$ representa o número de automóveis que passam
    por uma praça de pedágio no intervalo de 1 hora.

    Temos $X \in \{0, 1, 2, \cdots\}$.
\end{example}

\begin{example}\label{exp:ch02-va-lampada}
    $X$ representa o tempo de vida de uma lãmpada, em horas.

    Temos $X \in \left[0, +\infty\right)$.
\end{example}

\begin{example}[Continuação do \cref{exp:ch02-va-moeda}]
    Temos $X \in \{0, 1, 2\}$.
\end{example}

\begin{example}\label{exp:ch02-va-desoerdicio}
    $X$ representa o desperdício de refeição em um restaurante
    "bandejão", em gramas.

    Temos $X \in \{0\} \cup \left(0, +\infty\right)$.

    \begin{obs}
        A notação acima deixa o zero separado
        porque a variável aleatória $X$ tem um comportamento
        diferente neste valor.
    \end{obs}
\end{example}

\begin{definition}
    \begin{enumerate}
        \item \label{it:ch02-def-discretas}
        As variáveis aleatórias que assumem valores em
        um conjunto enumerável (finito ou infinito) são chamadas
        de \textbf{discretas}.
        \item \label{it:ch02-def-contínuas}
        As variáveis aleatórias que assumem valores em
        um intervalo são chamadas de \textbf{contínuas}.
    \end{enumerate}
\end{definition}

\begin{example}[Continuação do \cref{exp:ch02-va-moeda}]
    Calculamos:
    \begin{align*}
        \Prob(X = 0) &= \frac{1}{2} \cdot \frac{1}{2} = \frac{1}{4} \\
        \Prob(X = 1) &= \frac{1}{4} + \frac{1}{4} = \frac{1}{2} \\
        \Prob(X = 2) &= \frac{1}{4}
    \end{align*}
\end{example}

\begin{example}
    $X \in \{0, 1, 2, \cdots\}$
    \begin{align*}
        \Prob(X = x) &= \frac{\e^{-2}\cdot 2^x}{x!}
    \end{align*}

    Para atribuir probabilidades aos valores $x \in \{0, 1, 2, \cdots\}$,
    devemos ter $0 < \Prob(X = x) < 1$ e $\sum_{x=1}^n \Prob(X = x) = 1$.
\end{example}

Nos \cref{exp:ch02-va-moeda,exp:ch02-va-pedagio}, temos variáveis aleatórias
\textbf{discretas}.

No \cref{exp:ch02-va-lampada}, a variável aleatória é \textbf{contínua}.

No \cref{exp:ch02-va-desoerdicio}, a variável aleatória apresenta
comportamento de uma variável aleatória \textbf{discreta} quando $X \in \{0\}$
e \textbf{contínua} quando $X \in \left(0, +\infty\right)$. A este tipo de
compor tamento, dá-se o nome de \textbf{mistura} de variáveis aleatórias.
