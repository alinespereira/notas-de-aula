\section{Momentos}

\begin{definition}\label{def:ch02-momento}
    O \textbf{momento} de ordem $k$ de uma \va\ $X$
    é definido como
    \begin{align}
        \mu_k &= \Expected(X^k),\ k = 1,2,\cdots
        \label{eq:ch02-def-momento}
    \end{align}
    se $\Expected(X^k)$ existir ($\Expected(X^k) \in \R$).

    \begin{obs}
        O primeiro momento é a \textbf{média} da \va\ $X$.
        Para o primeiro momento, usamos a notação
        $\mu = \Expected(X)$.
    \end{obs}
\end{definition}

\begin{definition}
    A \textbf{variância} de uma \va\ $X$ é definida como
    \begin{align}
        \Var(X) &= \sigma^2 = \Expected\left[(X-\mu)^2\right]
        \label{eq:ch02-def-variancia}
    \end{align}

    Notar que $\Var(X) \in [0, +\infty)$.
\end{definition}

\begin{obs}
    Se $X = x_0$ (constante), então:
    \begin{align*}
        \Expected(X) &= x_0\text{ e} \\
        \Var(X) &= 0.
    \end{align*}
\end{obs}

\begin{definition}
    O \textbf{desvio-padrão} de uma \va\ $X$ é definido como
    $\sigma = \sqrt{\Var(X)}$.

    O desvio-padrão tem a mesma unidade da \va\ $X$.
\end{definition}

\begin{property}
    Se $X_1, X_2, \cdots, X_k$ são \va s e 
    $a_1, a_2, \cdots, a_k$ são constantes reais, então
    \begin{align}
        \Expected\left(\sum_{i=1}^k a_i \cdot X_i\right)
        &= \sum_{i=1}^k a_i \cdot \Expected(X_i)
        \label{eq:ch02-media-da-soma}
    \end{align}

    Em particular,
    \begin{align*}
        \Expected(X_1 + X_2) &= \Expected(X_1) + \Expected(X_2)
    \end{align*}
\end{property}

Funções de \va\ também são \va s. Como calcular $\Expected[g(X)]$?

\begin{result}
    \begin{align}
        \Expected[g(X)] &= \begin{cases}
            \sum_{x \in A} g(x)\cdot\Prob(X = x),
            & \text{se X é uma \va\ discreta} \\
            \\
            \int_{-\infty}^{+\infty} g(x)\cdot f(x) \wrt x,
            & \text{se X é uma \va\ contínua}
        \end{cases}
        \label{eq:ch02-espeanca-funcao-va}
    \end{align}
\end{result}

\begin{property}[Propriedades da variância]
    \label{prop:ch02-variancia}
    \begin{enumerate}
        \item \begin{align}
            \sigma^2 &= \Var(X) \notag \\
            &= \Expected(X^2) - \mu^2 \notag \\
            &= \Expected(X^2) - \left\{\Expected(X)\right\}^2
            \label{it:ch02-variancia-momentos}
        \end{align}
        \item Se $a_1$ e $a_2$ são constantes reais, então
        \begin{align}
            \Var(a_1 + a_2 X) &= a_2^2\Var(X)
            \label{it:ch02-variancia-constantes-reais}
        \end{align}
        e o desvio-padrão é $|a_2|\sqrt{\Var(X)}$.
    \end{enumerate}

    \begin{proof}
        \begin{enumerate}
            \item \begin{align*}
                \sigma^2 = \Var(X)
                &\overset{\text{\Cref{eq:ch02-def-variancia}}}{=}
                \Expected\left[(X-\mu)^2\right] \\
                &= \Expected\left[(X^2-2\mu X + \mu^2)\right] \\
                &\overset{\text{\Cref{eq:ch02-media-da-soma}}}{=}
                \Expected(X^2)
                    -\Expected(2\mu X)
                    + \Expected(\mu^2) \\
                &= \Expected(X^2)
                    - 2\mu\underbrace{\Expected(X)}_{=\mu}
                    + \mu^2 \\
                &= \Expected(X^2) - 2\mu^2 + \mu^2 \\
                &= \Expected(X^2) + \mu^2
            \end{align*}

            \item Temos \begin{align*}
                \Expected(a_1 + a_2 X)
                &\overset{\text{\Cref{eq:ch02-media-da-soma}}}{=}
                a_1 + a_2 \Expected(X) \\
                &= a_1 + a_2 \mu
            \end{align*}
            de modo que
            \begin{align*}
                \Var(a_1 + a_2 X)
                &\overset{\text{\Cref{eq:ch02-def-variancia}}}{=}
                \Expected\left\{
                    \left[
                        (a_1 + a_2 X)
                        - \Expected(a_1 + a_2 X)
                    \right]^2
                \right\} \\
                &= \Expected\left\{
                    \left[
                        (\cancel{a_1} + a_2 X)
                        - (\cancel{a_1} + a_2 \mu)
                    \right]^2
                \right\} \\
                &= \Expected\left\{
                    \left(a_2 X - a_2 \mu\right)^2
                \right\} \\
                &= \Expected\left\{
                    a_2^2 \left( X - a_2 \mu\right)
                \right\} \\
                &\overset{\text{\Cref{eq:ch02-media-da-soma}}}{=}
                a_2^2 \Expected\left\{
                    \left( X - a_2 \mu\right)
                \right\} \\
                &\overset{\text{\Cref{eq:ch02-def-variancia}}}{=}
                a_2^2 \underbrace{\Expected\left\{
                    \left( X - a_2 \mu\right)
                \right\}}_{=\Var(X)} \\
                &= a_2^2 \Var(X)
            \end{align*}
        \end{enumerate}
    \end{proof}
\end{property}

\begin{definition}
    O \textbf{coeficiente de variação} de uma \va\ $X$
    é definido como
    \begin{align}
        \frac{\sigma}{|\mu|},\ \text{se }\mu \neq 0.
    \end{align}

    Note que o coeficiente de variação não tem unidade.
\end{definition}

\begin{definition}
    A \textbf{função geradora de momentos} (\fgm)
    de uma \va\ X é definida como
    \begin{align}
        \phi(t) &= \Expected(\e^{tX}) \label{eq:ch02-fgm}
    \end{align}
    para $t \in \R$ tais que $\phi(t) \in (0, +\infty)$.
\end{definition}

\begin{result}[Relação entre momentos e \fgm]
    \begin{align}
        \mu_k &= \left.\diff[k]{}{t} \phi(t) \right|_{t=0}
    \end{align}
\end{result}

\begin{example}\label{exp:ch02-fgm-exponencial}
    A \va\ $X$ tem função densidade
    \begin{align*}
        f(x) &= \begin{cases}
            \lambda \e^{-\lambda x}, & \text{se } x > 0, \\
            0, & \text{se } x \leq 0,
        \end{cases}
    \end{align*}
    em que $\lambda$ é chamado \textit{parâmetro}
    dessa distribuição.

    Calcule a \fgm\ de $X$.

    \bigskip
    Pela \cref{eq:ch02-fgm}, temos que
    \begin{align*}
        \phi(t) &= \Expected(\e^{tX})
    \end{align*}
    em que o termo $\Expected(\e^{tX})$ pode ser
    calculado usando a \cref{eq:ch02-espeanca-funcao-va},
    observando que $X$ é uma \va\ contínua.
    Fazendo $g(X) = \e^{tX}$, temos:
    \begin{align*}
        \Expected(g(X))
        &= \int_{-\infty}^{+\infty} g(x) \cdot f(x) \wrt x \\
        \implies \phi(t) = \Expected(\e^{tX})
        &= \int_0^{+\infty} \e^{tx}
            \cdot \lambda \e^{-\lambda x} \wrt x \\
        &= \lambda \int_0^{+\infty}
                \e^{(t-\lambda) x} \wrt x \\
        &= \lambda \int_0^{+\infty}
                \e^{-(\lambda - t) x} \wrt x \\
        &= \frac{\lambda}{\lambda - t}
    \end{align*}
    Observe que para a integral convergir, devemos ter
    $\lambda - t > 0 \implies t < \lambda$.
    Além disso, para que $\phi(t) \in (0, +\infty)$,
    devemos ter também $\lambda > 0$.

    \bigskip
    Portanto:
    \begin{align*}
        \phi(t) = \Expected(\e^{tX})
        &= \frac{\lambda}{\lambda - t}
    \end{align*}
    para $0 < t < \lambda$.
\end{example}

Digamos que
\begin{align*}
    \phi(t) &= \frac{\sqrt{\lambda}}{\sqrt{\lambda} - t}
\end{align*}
Pergunta-se: qual a função densidade de $X$?

\begin{result}
    A \fgm\ \textbf{identifica} a distribuição da \va.
\end{result}

Comparando a \fgm\ dada com a do \cref{exp:ch02-fgm-exponencial},
concluímos que
    
\begin{align*}
    f(x) &= \begin{cases}
        \sqrt{\lambda} \e^{-\sqrt{\lambda} x}, 
        & \text{se } x > 0, \\
        0, & \text{se } x \leq 0,
    \end{cases}
\end{align*}

\begin{example}
    A \fgm\ de $X$ é dada por
    \begin{align*}
        \phi(t) = \Expected(\e^{tX})
        &= \frac{\lambda}{\lambda - t}
    \end{align*}

    Calcule $\mu$.

    \bigskip
    Temos que
    \begin{align*}
        \mu = \mu_1 &= \Expected(X) =  \Expected(X^1) \\
        &= \left.\diff{}{t}\phi(t)\right|_{t=0}
    \end{align*}
    em que
    \begin{align*}
        \diff{}{t}\phi(t) &= \frac{\lambda}{(\lambda - t)^2}
    \end{align*}
    
    Portanto
    \begin{align*}
        \mu &= \left.\diff{}{t}\phi(t)\right|_{t=0} \\
        &= \left.\frac{\lambda}{(\lambda - t)^2}\right|_{t=0} \\
        &= \frac{1}{\lambda} \\
    \end{align*}
\end{example}

\begin{example}
    $X$ é uma \va\ tal que $\Expected(X) < \infty$.
    Considere
    \begin{align*}
        h(a) &= \Expected[(X-a)^2]
    \end{align*}
    $a \in \R$.

    Determine o valor de $a$ que minimiza $h(a)$.

    \bigskip
    Escrevemos
    \begin{align*}
        h(a) &= \Expected[(X-a)^2] \\
        &= \Expected[X^2-2aX+a^2] \\
        &= \Expected(X^2)-2a\Expected(X)+a^2
    \end{align*}

    Calculamos
    \begin{align*}
        \diff{h(a)}{a} &= -2\Expected(X)+2a
    \end{align*}
    
    Igualando $\diff{h(a)}{a}$ a zero, obtemos
    \begin{align*}
        \diff{h(a)}{a} &= -2\Expected(X)+2a = 0 \\
        \therefore a &= \Expected(X)
    \end{align*}
    que é um ponto de mínimo global, pois
    \begin{align*}
        \diff[2]{h(a)}{a} &= 2 > 0
    \end{align*}

    Portanto, $a = \Expected(X)$.
\end{example}