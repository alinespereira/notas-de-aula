\section{Distribuição conjunta}

\begin{definition}[Função de probabilidade conjunta]
    \label{def:ch03-fp-2d}
    A \textbf{função de probabilidade} (ou
    \textbf{função massa de probabilidade}) de um vetor aleatório
    $(X, Y)$ é uma função que atribui probabilidades aos 
    valores de $X$ e $Y$ satisfazendo
    \begin{enumerate}
        \item $0 \leq \Prob(X = x, Y = y) \leq 1$, e
        \item $\sum_{x \in A_X} \sum_{y \in A_Y}
        \leq \Prob(X = x, Y = y) = 1$,
    \end{enumerate}
    sendo que $A_X$ e $A_Y$ são os conjuntos de valores
    de $X$ e $Y$.
    \begin{obs}
        $A_X$ e $A_Y$ são finitos ou infinitos enumeráveis.
    \end{obs}

    A função de probabilidade de $(X, Y)$ é chamada de
    \textbf{conjunta}.
\end{definition}

\begin{example}[Continuação do \Cref{exp:ch03-dois-dados}]
    Temos que
    \begin{align*}
        A_X &= \{2, 3, \cdots, 12\},\ \text{e} \\
        A_Y &= \{1, 2,\cdots, 6\}.
    \end{align*}

    A função de probabilidade conjunta do vetor $(X, Y)$ é:
    \begin{center}
        \begin{tabular}{cccccccccccc}
            \toprule
            \multirow{2}{*}{$Y$} & \multicolumn{11}{c}{$X$} \\
            \cmidrule{2-12}
             & 2               & 3 & 4 & 5 & 6 & 7 
                & 8 & 9 & 10 & 11 & 12 \\
            \midrule
            1 & $\sfrac{1}{36}$ & $\sfrac{2}{36}$ & $\sfrac{2}{36}$ 
            & $\sfrac{2}{36}$ & $\sfrac{2}{36}$ & $\sfrac{2}{36}$ 
            & $\cdot$ & $\cdot$ & $\cdot$ 
            & $\cdot$ & $\cdot$ \\
            2 & $\cdot$ & $\cdot$ & $\sfrac{1}{36}$
            & $\sfrac{2}{36}$ & $\sfrac{2}{36}$ & $\sfrac{2}{36}$
            & $\sfrac{2}{36}$ & $\cdot$ & $\cdot$
            & $\cdot$ & $\cdot$ \\
            3 & $\cdot$ & $\cdot$ & $\cdot$
            & $\cdot$ & $\sfrac{1}{36}$ & $\sfrac{2}{36}$
            & $\sfrac{2}{36}$ & $\sfrac{2}{36}$ 
            & $\cdot$ & $\cdot$ & $\cdot$ \\
            4 & $\cdot$ & $\cdot$ & $\cdot$ 
            & $\cdot$ & $\cdot$ & $\cdot$ 
            & $\sfrac{1}{36}$ & $\sfrac{2}{36}$
            & $\sfrac{2}{36}$ & $\cdot$ & $\cdot$ \\
            5 & $\cdot$ & $\cdot$ & $\cdot$ 
            & $\cdot$ & $\cdot$ & $\cdot$
            & $\cdot$ & $\cdot$ & $\sfrac{1}{36}$ 
            & $\sfrac{2}{36}$ & $\cdot$ \\
            6 & $\cdot$ & $\cdot$ & $\cdot$
            & $\cdot$ & $\cdot$ & $\cdot$
            & $\cdot$ & $\cdot$ & $\cdot$
            & $\cdot$ & $\sfrac{1}{36}$ \\
            \bottomrule
            & & & & & & & & & & Total & 1 \\
            \cmidrule{11-12}
        \end{tabular}
    \end{center}

    Na tabela, as entradas com probabilidade zero foram
    preenchidas por $\cdot$.
\end{example}

\begin{definition}[Função densidade conjunta]
    \label{def:ch03-fdp-2d}
    A \textbf{função densidade conjunta} é uma função 
    $f(x, y)$ satisfazendo:
    \begin{enumerate}
        \item $f(x, y) \geq 0$
        \item $\int_{-\infty}^{+\infty} \int_{-\infty}^{+\infty}
        f(x, y) \wrt x \wrt y = 1$
    \end{enumerate}
\end{definition}

Considere $B$ um subconjunto do $\R^2$:
\begin{enumerate}
    \item Se $(X, Y)$ é um vetor aleatório \textbf{contínuo}:
    \begin{align*}
        \Prob(B) &= {\iint}_B f(x, y) \wrt x \wrt y
    \end{align*}
    \item Se $(X, Y)$ é um vetor aleatório \textbf{discreto}:
    \begin{align*}
        \Prob(B) &= \underset{\substack{(x,y)\in B\\x \in A_X, y\in A_Y}}{\sum\sum}
        \Prob(X = x, Y = y)
    \end{align*}
\end{enumerate}

\begin{example}
    Considere o círculo de raio na origem e raio de comprimento $R$.
    Apresente uma função densidade que tenha valor igual a 0
    para pontos não pertencentes ao círculo.

    Temos
    \begin{align*}
        f(x, y) &= \begin{cases}
            \frac{1}{\pi R^2}, & \text{se } x^2 + y^2 \leq R^2 \\
            \\
            0, & \text{se } x^2 + y^2 > R^2
        \end{cases}
    \end{align*}

    \begin{center}
        \begin{tikzpicture}
            \begin{axis}[
                unbounded coords=jump,
                grid=none,
                axis x line=middle,
                axis y line=middle,
                xmin=-4, xmax=4,
                ymin=-4, ymax=4,
                xtick={3},
                ytick={3},
                xticklabels={$R$},
                yticklabels={$R$},
                xlabel={$x$},
                ylabel={$y$},
                x label style={anchor=west},
                y label style={anchor=south},
                x tick label style={anchor=north west},
                y tick label style={anchor=south east},
                legend style={fill=none,draw=none},
                unit vector ratio={1 1},
            ]

            \draw[thick] (0, 0) circle (3);

            \draw[ultra thick, blue,
                pattern=north west lines, pattern color=blue,
                ]
                (0, 0) ellipse (1 and 2);

            \node[blue, anchor=south east] at (-1, 0) {$B$};
            \end{axis}
        \end{tikzpicture}
    \end{center}

    \begin{align*}
        \Prob(B) &= \iint_B f(x, y) \wrt x \wrt y \\
        &= \iint_B \frac{1}{\pi R^2} \wrt x \wrt y \\
        &= \frac{1}{\pi R^2} \iint_B  \wrt x \wrt y \\
        &= \frac{1}{\pi R^2} \area(B) \\
        &= \frac{\area(B)}{\pi R^2}
    \end{align*}

    Note que a região $B$ está totalmente contida no círculo,
    portanto a função densidade é não nula em todos os pontos de $B$.
    Neste exemplo especificamente, a densidade é constante.
\end{example}

\begin{example}\label{exp:ch03-fdp}
    A função densidade de $(X, Y)$ é dada por
    \begin{align*}
        f(x, y) &= \begin{cases}
            x^2 + \frac{xy}{3}, 
                &\text{se } 0 \leq x \leq 1 \text{ e } 0 \leq y \leq 2, \\
            0, &\text{caso contrário}
        \end{cases}
    \end{align*}

    Note que $f(x, y) \ge 0$.
    Calculamos
    \begin{align*}
        \int_{-\infty}^{+\infty} \int_{-\infty}^{+\infty}
            f(x, y) \wrt x \wrt y
        &= \int_0^1 \int_0^2 x^2 + \frac{xy}{3} \wrt y \wrt x \\
        &= \int_0^1 \left[
            x^2y + \frac{xy^2}{6}
        \right]_{y=0}^{y=2} \wrt x \\
        &= \int_0^1 2x^2 + \frac{4x}{6} \wrt x \\
        &= \left[\frac{2x^3}{3} + \frac{4x^2}{12}\right]_{x=0}^{x=1} \\
        &= \frac{2}{3} + \frac{4}{12} \\
        &= \frac{2}{3} + \frac{1}{3} = 1
    \end{align*}
    Com isto, verificamos que $f(x, y)$ é de fato uma
    função densidade conjunta.

    Considere
    \begin{align*}
        B &= \{(x,y) : x + y \ge 1\}
    \end{align*}
    e calcule $\Prob(B)$.

    \begin{center}
        \begin{tikzpicture}
            \begin{axis}[
                unbounded coords=jump,
                grid=none,
                axis x line=middle,
                axis y line=middle,
                xmin=-2, xmax=2,
                ymin=-1.5, ymax=2.5,
                xtick={1},
                ytick={1, 2},
                xlabel={$x$},
                ylabel={$y$},
                x label style={anchor=west},
                y label style={anchor=south},
                legend style={fill=none,draw=none},
                unit vector ratio={1 1},
            ]

            \draw[name path=x] (0, 2) -- (1, 2);
            \draw[thick, fill=gray!5, opacity=0.5] (0, 0) rectangle (1, 2);

            \draw[name path=b, ultra thick, blue] (0,1) -- (1,0);

            \draw[blue, dotted, thick] (-1, 2) -- (2, -1);


            \tikzfillbetween[of=x and b]
                {pattern=north west lines, pattern color=blue};


            \node[blue] at (.55, 1.2) {$B$};
            \node at (.33, .33) {$\stcomp{B}$};
            \end{axis}
        \end{tikzpicture}
    \end{center}

    Usando diretamente a região $B$ como região de integração,
    temos:
    \begin{align*}
        \Prob(B) &= 
            \int_0^1
                \int_{1-x}^2 x^2 + \frac{xy}{3} \wrt y
            \wrt x \\
        &= \int_0^1
            \left[x^2 y + \frac{xy^2}{6}\right]_{y=1-x}^{y=2}
        \wrt x \\
        &= \int_0^1
            - x^2 (1 - x) + 2 x^2 - \frac{x (1 - x)^2}{6} + \frac{2 x}{3}
        \wrt x \\
        &= \int_0^1
            \frac{5 x^{3}}{6} + \frac{4 x^{2}}{3} + \frac{x}{2}
        \wrt x \\
        &= \left[
            \frac{5 x^{4}}{24} + \frac{4 x^{3}}{9} + \frac{x^{2}}{4}
        \right]_{x=0}^{x=1} \\
        &= \frac{65}{72}
    \end{align*}

    Outra possibilidade seria utilizarmos a probabilidade
    do evento complementar ao evento definido pela região $B$,
    e então usar o fato de que $\Prob(B) = 1 - \Prob(\stcomp{B})$.
    \begin{align*}
        \Prob(\stcomp{B}) &= 
            \int_0^1
                \int_0^{1-x} x^2 + \frac{xy}{3} \wrt y
            \wrt x \\
        &= \int_0^1
            \left[x^2 y + \frac{xy^2}{6}\right]_0^{1-x}
        \wrt x \\
        &= \int_0^1
            x^2 (1 - x) + \frac{x \left(1 - x\right)^{2}}{6}
        \wrt x \\
        &= \int_0^1
            - \frac{5 x^{3}}{6} + \frac{2 x^{2}}{3} + \frac{x}{6}
        \wrt x \\
        &= \left[
            - \frac{5 x^{4}}{24} + \frac{2 x^{3}}{9} + \frac{x^{2}}{12}
        \right]_{x=0}^{x=1} \\
        &= \frac{7}{72}
        \\
        \implies \Prob(B) &= 1 - \Prob(\stcomp{B}) \\
        &= 1 - \frac{7}{72} = \frac{65}{72}
    \end{align*}

    Como esperado, o mesmo resultado foi obtido em ambos os cálculos,
    apesar da segunda maneira ser pouco mais simples.
\end{example}

\begin{definition}
    A \textbf{função distribuição acumulada} do vetor $(X, Y)$
    é definida como
    \begin{align}
        F(x, y) &= \Prob(-\infty < X \le x, -\infty < Y \le y).
        \label{eq:ch03-fda-2d}
    \end{align}
    
    Se o vetor for contínuo:
    \begin{align}
        \diffp{F(x, y)}{xy} &= f(x, y) \label{eq:ch03-fdp-diff-fda}
    \end{align}
    e
    \begin{align}
        F(x, y) &= \int_{-\infty}^y \int_{-\infty}^x
            f(x, y) \wrt x \wrt y \label{eq:ch03-fda-int-fdp}
    \end{align}
\end{definition}
