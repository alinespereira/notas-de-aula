\section{Espaço amostral e evento}

\begin{definition}
    \textbf{Espaço amostral} de um expetimento aleatório é o conjunto de
    todos os possíveis resultados.
\end{definition}

\begin{notation}
    $S$ ou $\Omega$.
\end{notation}

\begin{example}\label{exp:ch01-espacoamostral-1}
    Lançamento de um dado de seis faces e observação da face
    exibida para cima.

    \begin{align*}
        S &= \{1,2,3,4,5,6\}
    \end{align*}
\end{example}

\begin{example}\label{exp:ch01-espacoamostral-2}
    Retirada de uma carta de um baralho de 52 cartas.

    \begin{align*}
        S &= \{1,2, \cdots ,52\}
    \end{align*}
\end{example}

\begin{example}\label{exp:ch01-espacoamostral-3}
    Ocorrência de chuva no dia seguinte.

    \begin{align*}
        S &= \{\text{sim}, \text{não}\} \\
        S &= \{1, 0\} \\
    \end{align*}
\end{example}

\begin{example}\label{exp:ch01-espacoamostral-4}
    Realzação de um sorteio da Mega Sena.

    \begin{align*}
        S = \{&(d_1, d_2, d_3, d_4, d_5, d_6): \\
            & d_j \in \{1, 2, \cdots, 60 \} \text{ para } j = 1, 2, \cdots, 6, \\
            & d_j \neq d_k \text{ para } j\neq k,\ 
                j = 1, 2, \cdots, 6,\ k = 1, 2, \cdots, 6
        \}
    \end{align*}
\end{example}

\begin{obs}
    Nos \cref{exp:ch01-espacoamostral-1,exp:ch01-espacoamostral-2,exp:ch01-espacoamostral-3,exp:ch01-espacoamostral-4}, o conjunto $S$ é \textit{finito}.
\end{obs}

\begin{example}\label{exp:ch01-espacoamostral-5}
    Uma moeda é lançada até que seja obtido o primeiro resultado "cara".
    
    Resultados: \begin{itemize}
        \item cara: $C$
        \item coroa: $\overline{C}$
    \end{itemize}
    
    Temos: \begin{align*}
        S &= \{C, \overline{C}C, \overline{C}\overline{C}C, \cdots \}
    \end{align*}

    \begin{obs}
        Note que o conjunto $S$ é \textit{infinito enumerável}.
    \end{obs}
\end{example}

\begin{example}\label{exp:ch01-espacoamostral-6}
    Uma lâmpada é deixada ligada até queimar, e o tempo é registrado.

    \begin{align*}
        S &= \left[0, +\infty\right)
    \end{align*}
\end{example}

\begin{example}\label{exp:ch01-espacoamostral-7}
    Resultado de um eletrocardiograma de um indivíduo
    em um intervalo de $T$ minutos.

    \begin{center}
        \begin{tikzpicture}
            \begin{axis}[
                unbounded coords=jump,
                grid=none,
                axis x line=middle,
                axis x line shift=.5,
                axis y line=middle,
                xmin=-2, xmax=25,
                ymin=-1, ymax=1.5,
                xtick={0, 40},
                ytick={-10, 10},
                xlabel={tempo},
                ylabel={sinal},
                y label style={anchor=south},
                x label style={anchor=west},
            ]
            \addplot+
            table[col sep=comma, x=t, y=signal] {chapters/eletro.dat};
            
            \addplot[mark=none, dashed, black] coordinates {(0, 0) (21, 0)};
            \addplot[mark=none, dashed, black] coordinates {(0, 1) (21, 1)};


            \node[anchor=north east] at (0,-0.5) {$O$};
            \node[anchor=north] at (20,-0.5) {$T$};
            
            \end{axis}
        \end{tikzpicture}
    \end{center}

    \begin{align*}
        S &= \{x: x \text{ é uma função positiva no intervalo } [0, T] \}
    \end{align*}

    Os elementos de $S$ são funções $x(t)$, $t \in [0, T]$.
\end{example}

\begin{example}\label{exp:ch01-espacoamostral-8}
    Resultado de um tiro a um alvo circular de raio igual a $R$

    \begin{center}
        \begin{tikzpicture}
            \begin{axis}[
                axis equal,
                unbounded coords=jump,
                grid=none,
                axis x line=middle,
                axis y line=middle,
                xmin=-1.25, xmax=1.25,
                ymin=-1.25, ymax=1.25,
                xtick={0, 40},
                ytick={-10, 10},
                xlabel=$x$,
                ylabel=$y$,
                y label style={anchor=south},
                x label style={anchor=west},
            ]
            \addplot+[only marks]
            table[col sep=comma] {chapters/target.dat};

            \pgfmathsetmacro{\R}{1}
            \coordinate (O) at (0, 0);
            \coordinate (P) at ({-\R*cos(45)}, {\R*sin(45)});

            \pgfmathsetmacro{\xA}{0.629203}
            \pgfmathsetmacro{\yA}{0.594032}
            \coordinate (A) at (\xA, \yA);
            \coordinate (PxA) at (\xA, 0);
            \coordinate (PyA) at (0, \yA);

            \draw (axis cs:0,0) circle [radius=\R];

            \draw[->, thin] (O) -- (P) {};
            \node[anchor=north east] at ($(O)!0.75!(P)$) {$R$};

            \addplot[dashed, red]
                coordinates {(0, 0.594032) (0.629203, 0.594032)};
            \addplot[dashed, red]
                coordinates {(0.629203, 0) (0.629203, 0.594032)};

            \addplot[
                only marks, 
                red,
                mark options={scale=2}
            ] coordinates {(\xA, \yA)};

            \node[anchor=north, red] at (PxA) {$x$};
            \node[anchor=east, red] at (PyA) {$y$};
            
            \end{axis}
        \end{tikzpicture}
    \end{center}

    \begin{align*}
        S = \{(x, y): 0 \leq x^2 + y^2 \leq R^2\}
    \end{align*}
\end{example}

\begin{definition}
    \textbf{Evento} (ou \textbf{evento aleatório}) é todo resultado ou 
    subconjunto de resultados
    de um experimento aleatório.

    Um \textbf{evento simples} é um evento com apenas um elemento de $S$
\end{definition}

\begin{example}\label{exp:ch01:evento-1}
    No \cref{exp:ch01-espacoamostral-1}, $\{5\}$ é um evento simples.
    "O resultado é um número par" não é um evento simples; é dado
    por $\{2, 4, 6\}$.
\end{example}