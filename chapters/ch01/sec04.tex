\section{Definições de probabilidade}

O conceito de probabilidade evoluiu desde a Idade Média.
Temos contribuições de Galileu, Pascal, Fermat, Cardano, Laplace,
Gauss, de Moivre, entre outros. Um dos conceitos iniciais é o
de "eventos igualmente possíveis".

\begin{definition}[Definição Clássica]
    Consideramos um espaço amostral $S$ com $N$ eventos simples
    igualmente possíveis. Seja $A$ um evento de $S$ composto por
    $m$ eventos simples. A probabilidade do evento $A$, denotada por
    $\Prob(A)$, é definida como
    \begin{align}\label{eq:ch01-def-classica}
        \Prob(A) &= \frac{m}{N}
    \end{align}
\end{definition}

\begin{example}
    Com uma única aposta de seis dezenas, qual é a probabilidade
    de acerto de uma quina em um sorteio da Mega Sena?
    
    \bigskip
    O expaço amostral $S$ já foi visto (\cref{exp:ch01-espacoamostral-4}).
    
    Temos que $N = \binom{60}{6}$.
    
    \medskip
    O evento A representa o acerto de uma quina (acerto de cinco
    das 6 dezenas sorteadas e erro da sexta dezena)

    Temos $m = \binom{6}{5} \cdot \binom{54}{1}$

    Logo:
    \begin{align*}
        \Prob(A)
            &= \frac{m}{N} \\
            &= \frac{\binom{6}{5} \cdot \binom{54}{1}}{\binom{60}{6}} \\
            &= \frac{6 \cdot 54}
                {50063860} \\
            &= 6.472 \cdot 10^{-6}
    \end{align*}
\end{example}

\begin{lemma}
    Nas condições da definição clássica, temos:
    \begin{enumerate}
        \item $\Prob(A) \geq 0$ para todo $A \in S$;
        \item Se $A$ e $B$ são mutuamente exclusivos, então
            $\Prob(A \cup B) = \Prob(A) + \Prob(B)$; e
        \item $\Prob(S) = 1$
    \end{enumerate}
\end{lemma}

\begin{definition}[Definição Fequentista]
    O experimento é repetido $n$ vezes nas mesmas condições.
    Seja $n(A)$ o número de vezes em que o event $A$ ocorre
    nas n repetições.

    Definimos a frequência relativa do evento $A$, denotada
    por $f_{n, A}$, como
    \begin{align}
        f_{n, A} &= \frac{n(A)}{n} \label{eq:ch01-def-frequentista}
    \end{align}
\end{definition}

\begin{example}
    Buffon (séc. \romanum{18}) realizou 4040 lançamentos de uma moeda
    e observou a ocorrência de 2048 resultados "cara". Assim:
    \begin{align*}
        f_{n, A} &= \frac{n(A)}{n} \\
        &= \frac{2048}{4040} \\
        &= 0.5069
    \end{align*}
\end{example}

Quando $n \to \infty$, $f_{n, A}$ deve estabilizar em torno de
um número.

Para valores "grandes"\ de $n$, a probabilidade do evento $A$ é
aproximadamente dada por $f_{n, A}$.

\begin{lemma}
    A frequência relativa $f_{n, A}$ tem as seguintes propriedades:
    \begin{enumerate}
        \item para todo evento $A$, $0 \leq f_{n, A} \leq 1$;
        \item se $A$ e $B$ são dois eventos de $S$ mutuamente exclusivos,
            então $f_{n, A \cup B} = f_{n,A} + f_{n,B}$; e
        \item $f_{n,S} = 1$.
    \end{enumerate}
\end{lemma}

\begin{definition}[Definição Subjetiva]
    A probabilidade de um evento é a medida da crença que o indivíduo
    tem sobre a ocorrência do evento.

    Diferentes indivíduos podem atribuir probabilidades diferentes
    para um mesmo evento.
\end{definition}

\begin{definition}[Definição Axiomática]
    formulada por Kolmogorov (1931).

    $S$ denota o espaço amostral do experimento.
    
    $\F$ denota uma classe de conjuntos (eventos) de $S$.

    \begin{obs}
        Nem sempre a classe $\F$ representa todos os possíveis subconjuntos
        (ou eventos) de $S$, que é chamado de conjunto das partes de $S$,
        denotado por $\Power(S)$.

        Se $S$ é um intervalo, a classe $\F$ \textbf{não} é $\Power(S)$.
        Neste caso, $\Power(S)$ é um conjunto infinito não enumerável.
    \end{obs}
\end{definition}

\begin{definition}\label{def:axiomatica}
    Probabilidade é uma função definida nos elementos da classe $\F$
    satisfazendo:
    \begin{enumerate}
        \item \label{it:ch01-axioma-1} $\Prob(A) \geq 0$ para todo $A \in \F$;
        \item \label{it:ch01-axioma-2} Se $A_1, A_2, \cdots$ é uma sequência de eventos
        mutuamente exclusivos de $\F$, então \begin{align*}
            \Prob\left(\bigcup_{i=1}^\infty A_i\right)
                &= \sum_{i=1}^\infty \Prob(A_i);
        \end{align*} e
        \item \label{it:ch01-axioma-3}$\Prob(S) = 1$.
    \end{enumerate}
\end{definition}

\begin{obs}
    As definições clássica e frequentista obedecem à definição axiomática.
\end{obs}

Se $S$ for enumerável (finito ou infinito), a classe $\F$ é dada por
$\F = \Power(S)$. Representamos $S$ por $S = \{\omega_1, \omega_2, \cdots\}$.
Associamos a cada $\omega_i \in S$ uma probabilidade $\Prob(\omega_i) \geq 0$
e $\sum_{i = 1}^\infty \Prob(\omega_i) = 1 = \Prob(S)$. Neste caso, definimos
\begin{align*}
    \Prob(A) &= \sum_{i: \omega_i \in A} \Prob(\omega_i) \\
    &= \sum_{\omega \in A} \Prob(\omega)
\end{align*}

Esta definição está de acodo com a definição axiomática.