\section{Operações entre eventos}

\begin{definition}
    A união de dois eventos $A$ e $B$ ($A \cup B$) é o evento
    que ocorre se pelo menos um dos eventos ocorrer.
\end{definition}

\begin{definition}
    A interseção de dois eventos $A$ e $B$ ($A \cap B$) é o evento
    que ocorre se ambos ocorrerem.
\end{definition}

\begin{definition}
    O complemento do evento $A$ ($\stcomp{A}$) é o evento que ocorre
    quando $A$ não ocorrer.
\end{definition}

\begin{obs}
    $\{x\}$ é o resultado de um experimento aleatório.
    Se o evento $A$ ocorreu, então $x \in A$.
\end{obs}

Se a ocorrência do evento $A$ impede a ocorrência do evento $B$,
dizemos que $A$ e $B$ são mutuamente exclusivos:
\begin{align*}
    A \cap B &= \emptyset.
\end{align*}

\begin{example}
    $A$ e $\stcomp{A}$ são mutuamente exclusivos:
    \begin{align*}
        A \cap \stcomp{A} &= \emptyset.
    \end{align*}
\end{example}

\begin{lemma}\label{eq:ch01-ops}
    $A$, $B$ e $C$ são eventos de um espaço amostal S. Temos que:
    \begin{align}
        (A \cup B) \cap C &= (A \cap C) \cup (B \cap C) 
            \label{eq:ch01-ops-1} \\
        (A \cap B) \cup C &= (A \cup C) \cap (B \cup C) 
            \label{eq:ch01-ops-2} \\
        \stcomp{(A \cup B)} &= \stcomp{A} \cap \stcomp{B}
            \label{eq:ch01-ops-3} \\
        \stcomp{(A \cap B)} &= \stcomp{A} \cup \stcomp{B}
            \label{eq:ch01-ops-4}
    \end{align}
\end{lemma}

\begin{definition}
    O evento $\bigcup_{i = 0}^\infty A_i$ é o evento que ocorre
    quando \textit{pelo menos um} dos eventos $A_i$ ocorrer.
\end{definition}

\begin{definition}
    O evento $\bigcap_{i = 0}^\infty A_i$ é o evento que ocorre
    quando \textit{todos} os eventos $A_i$ ocorrerem.
\end{definition}
