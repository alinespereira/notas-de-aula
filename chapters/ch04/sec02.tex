\section{Uma variável aleatória discreta}

$X$ é uma \va\ discreta com valores $x_1,\ x_2,\ \cdots$.

Considere a transformação $Y = H(X)$. Com a transformação,
obtemos $H(x_1)$, $H(x_2)$, $\cdots$, sendo que podem ocorrer
valores de $Y$ repetidos (dependendo da função $H$).

Para cada valor de $X$, temos as probabilidades $p_X(x) = \Prob(X = x)$
(ou seja, conhecemos a distribuição de $X$).

O objetivo é obter $p_Y(y) = \Prob(Y = y)$. Para tal, calculamos:
\begin{align}
    p_Y(y) = \Prob(Y = y)
    &= \sum_{x: H(x) = y} P(X = x)
    \label{eq:ch04-trans-discreta}
\end{align}

\begin{example}\label{exp:ch04-trans-discreta-1}
    $X$ assume valores no conjunto $\Nat = \{1, 2, \cdots \}$ com
    \begin{align*}
        \Prob(X = x) &= \frac{1}{2^x},\ x \in \Nat
    \end{align*}

    Definimos
    \begin{align*}
        Y &= \begin{cases}
            1  &,\ \text{se $X$ for par} \\
            -1 &,\ \text{se $X$ for ímpar}
        \end{cases}
    \end{align*}

    Calculamos
    \begin{align*}
        \Prob(Y = 1) &= \sum_{x: H(x) = 1} P(X = x) \\
        &= \sum_{x: x \text{ é par}} P(X = x) \\
        &= \frac{1}{2^2} + \frac{1}{2^4} + \frac{1}{2^6} + \cdots \\
        &= \frac{\frac{1}{2^2}}{1 - \frac{1}{2^2}} \\
        &= \frac{\frac{1}{4}}{\frac{3}{4}} = \frac{1}{3}
    \end{align*}
    e
    \begin{align*}
        \Prob(Y = -1) &= \sum_{x: H(x) = -1} P(X = x) \\
        &= \sum_{x: x \text{ é impar}} P(X = x) \\
        &= \frac{1}{2^1} + \frac{1}{2^3} + \frac{1}{2^5} + \cdots \\
        &= \frac{\frac{1}{2^1}}{1 - \frac{1}{2^2}} \\
        &= \frac{\frac{1}{2}}{\frac{3}{4}} = \frac{2}{3}
    \end{align*}
    ou, alternativamente,
    \begin{align*}
        \Prob(Y = -1) &= 1 - \Prob(Y = 1) \\
        &= 1 - \frac{1}{3} = \frac{2}{3}
    \end{align*}
\end{example}

\begin{example}\label{exp:ch04-trans-discreta-2}
    A \va\ $X$ tem função densidade $f_X(x) = \e^{-x}$, se $x > 0$.
    Definimos
    \begin{align*}
        Y &= \begin{cases}
            0 &,\ \text{se } x \le \sfrac{1}{2} \\
            1 &,\ \text{se } x > \sfrac{1}{2}
        \end{cases}
    \end{align*}

     Desejamos obter a distribuição de $Y$. Como $Y \in \{0, 1\}$,
     isto equivale a dizer que desejamos calcular $\Prob(Y = 0)$
     e $\Prob(Y = 1)$.

     Calculamos
    \begin{align*}
        \Prob(Y = 0) &= \Prob\left(0 < X \le \frac{1}{2}\right) \\
        &= F_X\left(\frac{1}{2}\right)
    \end{align*}
    em que
    \begin{align*}
        F_X(x) &= \Prob(0 < X \le x) \\
        &= \int_0^{x} f_X(u)\wrt u \\
        &= \int_0^{x} \e^{-u} \wrt u \\
        &= \left.-\e^{-u}\right|_{u=0}^{u=x} \\
        &= (-\e^{-x}) - (-\e^{-0}) \\
        &= 1 - \e^{-x}
    \end{align*}
    para $x > 0$. Deste modo:
    \begin{align*}
        \Prob(Y = 0) &= F_X\left(\frac{1}{2}\right) \\
        &= 1 - \e^{-\frac{1}{2}} \approx 0.3935
    \end{align*}
    e
    \begin{align*}
        \Prob(Y = 1) &= 1 - \Prob(Y = 0) \\
        &= 1 - (1 - \e^{-\frac{1}{2}}) \\
        &= \e^{-\frac{1}{2}} \approx 0.6065
    \end{align*}

    Outra maneira de calcular $\Prob(Y = 1)$ é:
    \begin{align*}
        \Prob(Y = 1) &= \Prob\left(X > \frac{1}{2}\right) \\
        &= 1 - \Prob\left(0 < X \le \frac{1}{2}\right) \\
        &= 1 - F_X\left(\frac{1}{2}\right) \\
        &= 1 - (1 - \e^{-\frac{1}{2}}) \\
        &= \e^{-\frac{1}{2}} \approx 0.6065
    \end{align*}
\end{example}

O procedimento dos \cref{exp:ch04-trans-discreta-1,exp:ch04-trans-discreta-2} pode ser generalizado para uma quantidade qualquer
de intervalos de $X$.