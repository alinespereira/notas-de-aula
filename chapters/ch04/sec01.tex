\section{Motivação}

\begin{example}\label{exp:ch04-fdp-area}
    A \va\ $X$ representa o raio de um círculo , sendo que $0 < X < R$.
    A função densidade é $f_X(x)$, sendo que $0 < x < R$.
    $Y$ é a área do círculo, que é uma \va, e é dada por $Y = \pi X^2$,
    sendo que $0 < Y < \pi R^2$.

    \bigskip
    Qual é a função densidade de $Y$?
\end{example}

\begin{example}\label{exp:ch04-fdp-gasto-per-capita}
    O vetor aleatório $(X, Y)$ tem função densidade $f_{X, Y}(x, y)$,
    se $x > 0$ e $y > 0$.
    $X$ representa a renda familiar mensal e $Y$ representa o
    gasto familiar mensal.

    Definimos $Z = \frac{X}{Y}$ (que tem interpretação).

    \bigskip
    Qual é a função densidade da variável $Z$?
\end{example}