\renewcommand\qedsymbol{$\blacksquare$}
\allowdisplaybreaks

\everymath{\displaystyle}
\DeclarePairedDelimiter\ceil{\lceil}{\rceil}
\DeclarePairedDelimiter\floor{\lfloor}{\rfloor}

% Euler's number
\DeclareMathOperator{\e}{e}
\DeclareMathOperator{\R}{\mathbb{R}}
\DeclareMathOperator{\Nat}{\mathbb{N}}
\DeclareMathOperator{\Q}{\mathbb{Q}}
\DeclareMathOperator{\Z}{\mathbb{Z}}
\DeclareMathOperator{\C}{\mathbb{C}}

\DeclareMathOperator{\sen}{sen}

\DeclareMathOperator{\area}{\acute{a}rea}
% Indicator function
\DeclareMathOperator{\Ind}{\mathbb{I}}

\newcommand{\wrt}{\,\mathrm{d}}

\let\upto\nearrow
\let\downto\searrow
\newcommand{\verteq}{\rotatebox{90}{$\,=$}}

% \renewenvironment{proof}{{\small{\bfseries Demonstração.}}}{}
\renewcommand\qedsymbol{$\blacksquare$}
\let\iff\Leftrightarrow

% Reduce fractions
\usepackage{xintgcd,xintfrac}
\newcommand*\reducedfrac[2]{
    \begingroup
        \pgfmathparse{#1}
        \edef\p{\pgfmathresult}
        \pgfmathparse{#2}
        \edef\q{\pgfmathresult}
        \edef\gcd{\xintGCD{\p}{\q}}%
        \frac{\xintNum{\xintDiv{\p}{\gcd}}}{\xintNum{\xintDiv{\q}{\gcd}}}%
    \endgroup
}

% Binomial number calculator
\usepackage{xintexpr}
\newcommand*\binomeval[2]{
    \xinttheexpr subs(subs(x!//y!//(x-y)!,y=#2), x=#1)\relax
}

\newcommand*\evaltoprec[2]{
    \pgfkeys{/pgf/fpu}
    \pgfmathparse{#1}
    \pgfmathprintnumberto[fixed,precision=#2]{\pgfmathresult}{\value}
    \value
    \pgfkeys{/pgf/fpu=false}
}

\newcommand*\evaltoprecsci[2]{
    \pgfkeys{/pgf/fpu}
    \pgfmathparse{#1}
    \pgfmathprintnumberto[sci,precision=#2]{\pgfmathresult}{\value}
    \value
    \pgfkeys{/pgf/fpu=false}
}

\newcommand{\tikzarc}[1]{%
    \tikzmarknode{a}{#1}
    \begin{tikzpicture}[overlay,remember picture]
        \draw ([yshift=1pt]a.north west)
            to[bend left=20] ([yshift=1pt]a.north east);
    \end{tikzpicture}%
}